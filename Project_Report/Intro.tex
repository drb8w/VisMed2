\section{Motivation}

In the STAR report we have discussed various techniques related to the visualization of cardiovascular structures. We have covered both model-based and model-free approaches. Model-based mesh generation follows some more restrictive assumptions about the vascular structure whereas model-free methods such as direct volume rendering tries to capture as much original information as possible. Model-based techniques are often used for treatment planning and basic anatomical overview, whereas model-free methods are required for precise diagnostics.

We decided that for the project implementation we wanted to focus on direct volume rendering. The goal was to effectively visualize thin vessel structures and if possible even their narrowings due to stenoses etc. At first we considered to implement the automatic transfer function specification for visual emphasis of coronary artery plaque by Glasser et al.~\cite{glasser2010automatic}. However we realized that due to time constraints this would not be possible as it involves a large number of different pipelines stages. We thus decided to implement instant volume visualization using \emph{Maximum Intensity Difference Accumulation} (MIDA) by Bruckner and Gr{\"o}ller~\cite{bruckner2009instant}.


% \section{Related Work}