\section{Skeletonization}

Skeletonization can be seen either as integral part of surface visualization techniques or as an separate preprocessing step. Actually both is true according to the observed methods as some surface visualization techniques compute the skeleton in an internal stage of the process. Here we focus on the preprocessing step.
Skeletonization here is actually a reduction of dimension on the segmentation data. We actually reduce the volumetric segmentation data or two dimensional surface data to a one dimensional graph structure. Nevertheless the graph itself is embedded in a 3D space.
The skeleton itself is a compact representation of 2D or 3D shapes tha preserves many of the topological and size characteristics \cite{ebert2002augmented}. Bium et al.~\cite{bium1964transformation} provides a common definition of the skeleton as the locus of centers of maximal disks (or spheres) contained in the original object.

There are actually three ways of genearting skeletons and medial axes. The first one is \emph{morphological thinning} there the boundary of an object is peeled off layer by layer till those points remain which removal would cause topological change \cite{ebert2002augmented}. As this is based on heuristics and is not based on the maximal disks approach this is the weakest of the skeletonization techniques, though the easiest one. The second is based on \emph{Voronoi diagrams} where the diagram represent the boundary's medial axis \cite{ebert2002augmented}. While the most accurate one, this technique is also the most complex and expensive one. The third way is based on \emph{distance transforms} of oebject's boundaries \cite{ebert2002augmented}. Here Sethian et al.~\cite{sethian1996fast} proposed the \emph{Fast Marching Method} for evolution of boundaries in normal direction where the skeleton lies along the singularites. However the detection of those singularities is difficult and possibly unstable.  

% reduction of one dimension 