\subsection{Conclusions}

We have reviewed some recent techniques in the field of direct and indirect (model-free and model-based) visualization of vascular structures. 

Regarding model-free techniques the focus was on improvements to the traditional direct volume rendering methodology. Specifically, improvements to the design of transfer functions were presented, that provide both depth information as well as highlighting high intensity structures such as contrast-enhanced vessels or vessel abnormalities. An essential usability feature is the provision of transfer functions that require little or no user interaction while still providing good results for arbitrary data.
Regarding projections to 2D views, improvements to traditional Curved Planar Reformation have been presented that remove the need of rotating around a vessel by aggregating intensities to provide a rotation independent overview.

With regard to the indirect surface visualization techniques we saw that the proper choice of a method depends heavily on the intended use case. For educational visualization sometimes simple model-based visualizations are preferred as they remove unnecessary or abnormal data and focus on the statistical average model to provide the basic overview to students. In clinical diagnosis however, every abnormal detail of a blood vessel is likely of high interest, meaning that model-free approaches are often required as they more closely represent the actual data. Also, potential subsequent use of the generated models, such as in fluid simulations, may influence the selected methodology, e.g. to have quad based patches instead of triangular ones for mesh description.

To conclude, we should emphasize that the choice of the visualization method should be motivated primarily by the application requirements.
The field of vascular visualization is still evolving and provides numerous research challenges, especially in the automatization of visualization techniques for arbitrary data, especially for detection of pathological conditions, the combination of both context and detail rendering, as well as uncertainty visualization in diagnostic settings. An interesting topic of research would also be the tracking of surgical instruments during an operation and their visualization in the context of direct volume rendering.